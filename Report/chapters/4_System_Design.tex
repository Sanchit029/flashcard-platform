This chapter presents the comprehensive system design for StudyGenie, including architectural decisions, component design, database schema, and user interface design. The design follows modern software engineering principles and best practices to ensure scalability, maintainability, and user experience.

\section{System Architecture}

\subsection{Overall Architecture}
StudyGenie follows a three-tier architecture pattern consisting of:
\begin{itemize}
    \item \textbf{Presentation Tier:} React.js frontend with responsive user interface
    \item \textbf{Application Tier:} Node.js with Express.js backend handling business logic
    \item \textbf{Data Tier:} MongoDB database for persistent data storage
\end{itemize}

\subsection{MERN Stack Architecture}
The system utilizes the MERN (MongoDB, Express.js, React.js, Node.js) stack:

\begin{itemize}
    \item \textbf{MongoDB:} NoSQL document database for flexible data storage
    \item \textbf{Express.js:} Web application framework for Node.js
    \item \textbf{React.js:} Frontend library for building interactive user interfaces
    \item \textbf{Node.js:} JavaScript runtime for backend development
\end{itemize}

\begin{figure}[H]
\centering
\includegraphics[width=0.9\textwidth]{chapters/System_Arch_4.1_.png}
\caption{StudyGenie System Architecture Overview}
\label{fig:system_architecture}
\end{figure}

\subsection{Microservices Design Patterns}
The backend is organized into logical modules:
\begin{itemize}
    \item \textbf{Authentication Service:} User registration, login, and session management
    \item \textbf{Content Processing Service:} Text extraction and AI processing
    \item \textbf{Study Material Service:} CRUD operations for flashcards and quizzes
    \item \textbf{Analytics Service:} Progress tracking and performance analytics
\end{itemize}

\section{Database Design}

\subsection{Data Model}
The system uses a document-based data model with the following primary collections:

\textbf{Core Database Collections:}
\begin{itemize}
    \item \textbf{User Collection:} Stores user authentication data including username, email, hashed password, and timestamps
    \item \textbf{FlashcardSet Collection:} Contains generated study materials with user association, content type, questions, and metadata
    \item \textbf{Document Collection:} Manages uploaded documents with extracted content, generated materials, and processing results
\end{itemize}

\textbf{Schema Design Principles:}
\begin{itemize}
    \item Document-based structure optimized for MongoDB
    \item Embedded documents for related data (questions within sets)
    \item Foreign key references for user associations
    \item Flexible schema supporting multiple content types
    \item Indexed fields for performance optimization
\end{itemize}

\begin{figure}[H]
\centering
\includegraphics[width=0.8\textwidth]{chapters/DataBase_Schema_4.2_.png}
\caption{Database Schema and Relationships}
\label{fig:database_schema}
\end{figure}

\subsection{Database Relationships}
\begin{itemize}
    \item One-to-Many relationship between User and FlashcardSet
    \item One-to-Many relationship between User and Document
    \item Embedded documents for questions and flashcards within their parent documents
\end{itemize}

\section{API Design}

\subsection{RESTful API Architecture}
The system implements a RESTful API following standard HTTP conventions:

\textbf{Authentication Endpoints:}
\begin{itemize}
    \item \texttt{POST /api/auth/register} - User registration
    \item \texttt{POST /api/auth/login} - User authentication
\end{itemize}

\textbf{Content Processing Endpoints:}
\begin{itemize}
    \item \texttt{POST /api/ai/upload/text} - Process text content
    \item \texttt{POST /api/ai/upload/pdf} - Process PDF documents
    \item \texttt{POST /api/ai/generate-flashcards} - Generate Q/A flashcards
    \item \texttt{POST /api/ai/generate-mcqs} - Generate MCQ questions
\end{itemize}

\textbf{Flashcard Management Endpoints:}
\begin{itemize}
    \item \texttt{GET /api/flashcard-sets} - Retrieve user's flashcard sets
    \item \texttt{POST /api/flashcard-sets} - Create new flashcard set
    \item \texttt{GET /api/flashcard-sets/:id} - Get specific flashcard set
    \item \texttt{PUT /api/flashcard-sets/:id} - Update flashcard set
    \item \texttt{DELETE /api/flashcard-sets/:id} - Delete flashcard set
\end{itemize}

\subsection{API Security}
\begin{itemize}
    \item JWT (JSON Web Token) based authentication
    \item Middleware for route protection
    \item Input validation and sanitization
    \item CORS (Cross-Origin Resource Sharing) configuration
\end{itemize}

\section{Frontend Design}

\subsection{Component Architecture}
The React frontend follows a component-based architecture:

\textbf{Layout Components:}
\begin{itemize}
    \item \texttt{App.jsx} - Main application component with routing
    \item \texttt{Navbar.jsx} - Navigation bar with authentication controls
\end{itemize}

\textbf{Authentication Components:}
\begin{itemize}
    \item \texttt{Login.jsx} - User login form
    \item \texttt{Register.jsx} - User registration form
\end{itemize}

\textbf{Core Feature Components:}
\begin{itemize}
    \item \texttt{Dashboard.jsx} - Main user dashboard
    \item \texttt{Upload.jsx} - Content upload interface
    \item \texttt{FlashcardSet.jsx} - Interactive flashcard display
    \item \texttt{Quiz.jsx} - MCQ quiz interface
    \item \texttt{StudyCards.jsx} - Study session management
\end{itemize}

\textbf{Utility Components:}
\begin{itemize}
    \item \texttt{Flashcard.jsx} - Individual flashcard with flip animation
    \item \texttt{ComingSoonAnalytics.jsx} - Analytics placeholder
\end{itemize}

\subsection{State Management}
\begin{itemize}
    \item React Hooks for local component state
    \item Context API for authentication state
    \item Axios for HTTP requests and API communication
\end{itemize}

\subsection{Routing}
React Router for client-side navigation:
\begin{itemize}
    \item Protected routes for authenticated users
    \item Dynamic routing for flashcard sets and quizzes
    \item Automatic redirection based on authentication status
\end{itemize}

\section{User Interface Design}

\subsection{Design Principles}
\begin{itemize}
    \item \textbf{Responsive Design:} Mobile-first approach using Tailwind CSS
    \item \textbf{Accessibility:} WCAG guidelines compliance
    \item \textbf{Usability:} Intuitive navigation and clear visual hierarchy
    \item \textbf{Performance:} Optimized loading and interactive elements
\end{itemize}

\subsection{Visual Design}
\begin{itemize}
    \item \textbf{Color Scheme:} Professional blue and gray palette
    \item \textbf{Typography:} Clean, readable fonts with proper hierarchy
    \item \textbf{Layout:} Card-based design with consistent spacing
    \item \textbf{Animations:} 3D flip animations for flashcards, smooth transitions
\end{itemize}

\subsection{User Experience Flow}
\begin{enumerate}
    \item User registration/login
    \item Content upload (text or PDF)
    \item AI processing and material generation
    \item Interactive study session
    \item Progress tracking and analytics
\end{enumerate}

\section{AI Processing Pipeline}

\subsection{Content Processing Workflow}
\begin{enumerate}
    \item Input validation and preprocessing
    \item Text extraction (for PDFs)
    \item Content analysis and concept identification
    \item AI-powered generation (flashcards, MCQs, summaries)
    \item Post-processing and quality validation
    \item Storage in database
\end{enumerate}

\subsection{AI Service Integration}
\begin{itemize}
    \item Modular AI service architecture
    \item Error handling and fallback mechanisms
    \item Caching for improved performance
    \item Rate limiting for API usage management
\end{itemize}

\section{Security Design}

\subsection{Authentication and Authorization}
\begin{itemize}
    \item JWT token-based authentication
    \item Secure password hashing using bcrypt
    \item Session management and token expiration
    \item Role-based access control (future enhancement)
\end{itemize}

\subsection{Data Protection}
\begin{itemize}
    \item HTTPS encryption for data transmission
    \item Input validation and sanitization
    \item SQL injection prevention
    \item XSS (Cross-Site Scripting) protection
\end{itemize}

\section{Performance Considerations}

\subsection{Frontend Optimization}
\begin{itemize}
    \item Code splitting and lazy loading
    \item Optimized bundle sizes
    \item Caching strategies for static assets
    \item Responsive image loading
\end{itemize}

\subsection{Backend Optimization}
\begin{itemize}
    \item Database indexing for improved query performance
    \item API response caching
    \item Asynchronous processing for AI operations
    \item Connection pooling for database operations
\end{itemize}

This comprehensive system design ensures that StudyGenie is built with scalability, maintainability, and user experience as core considerations, providing a solid foundation for the intelligent study material generation system.